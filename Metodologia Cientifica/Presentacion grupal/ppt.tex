\documentclass{beamer}
\usecolortheme{dolphin}
\usetheme{Antibes}


\begin{document}
	

	\begin{frame}
		\frametitle{Vida marina en Lanzarote}
		
		La diversidad y cantidad de cetáceos que habitan las aguas que rodean las islas de Lanzarote y Fuerteventura las han convertido en el foco de una campaña impulsada por WWF para crear un \textbf{SANTUARIO DE CETÁCEOS}. El objetivo es garantizar la preservación de la zona para la conservación de las especies y su hábitat.

		\pause
		
		\begin{block}{¡Una de las islas con mayor diversidad!}
			Se tienen registros de al menos 29 especies de cetáceos, entre delfines, zifios, calderones, cachalotes…, que viven o pasan en su migración por las aguas que rodean estas islas.
		\end{block}
		
	\end{frame}
	
	\begin{frame}
		\frametitle{Vida marina en Lanzarote}
		
		\begin{alertblock}{¿Te apetece bucear?}
			Lanzarote tiene algunos de los mejores fondos marinos de Europa, la formación rocosa y volcánica de las islas Canarias hacen que el mundo submarino de la isla sea increíble. La visibilidad que alcanza unos 40 metros, las temperaturas subtropicales y la abundancia y variedad de especies marinas te asombrarán.
		\end{alertblock}
		
	\end{frame}


\begin{frame}
	\frametitle{Puntos de buceo}
	
	 La isla se divide en 4 zonas de buceo: 
	
	\begin{columns}
		\begin{column}{0.48\textwidth}
			\begin{itemize}
				\item \onslide<1->{\textbf{Playa Blanca}: es una de las zonas más tranquilas para el buceo, no se superan profundidades de 18 metros y está protegida de los vientos alisios del norte.}
			\end{itemize}
		\end{column}
		
		\begin{column}{0.48\textwidth}
			\begin{itemize}
				\item \onslide<2->{\textbf{Puerto del Carmen}: en el centro este de la isla, bajo Arrecife, es probablemente la zona más concurrida de la isla para bucear. Las inmersiones se realizan a lo largo de su veril, una impresionante pared que empieza en 15 metros y desciende hasta los 40.}
			\end{itemize}
		\end{column}
	\end{columns}
	
	
\end{frame}

\begin{frame}
	\frametitle{Puntos de buceo}
	
	La isla se divide en 4 zonas de buceo: 
	
	\begin{columns}
		\begin{column}{0.48\textwidth}
			\begin{itemize}
				\item \onslide<1->{\textbf{Mala}: se encuentra al noroeste de la isla, junto a los pueblos de Arrieta y Charco del Palo. Es una de nuestros sitios de buceo preferidos por sus paisajes submarinos. Impresionantes pendientes de arena blanca que se pierden en el azul cortadas perpendicularmente por lenguas volcánicas que descienden hacia las profundidades.}
			\end{itemize}
		\end{column}
	
		
		\begin{column}{0.48\textwidth}
			\begin{itemize}
				\item \onslide<2->{\textbf{La Graciosa}: es un espacio natural protegido y constituye la reserva marina más grande de Europa.}
			\end{itemize}
		\end{column}
	\end{columns}
	
	
\end{frame}

\end{document}
