\documentclass[a4paper, 11pt]{article}  % Tipo de documento (report para TFM)
\usepackage{amsmath, amsthm, amssymb} %Ecuaciones, teoremas y símbolos matemáticos
\usepackage[utf8]{inputenc}
\usepackage[spanish]{babel}
\usepackage{graphicx}

\graphicspath{ {./images/} }
\title{Mi primer documento} %Preámbulo
\author{Alfonso \and Pepe \thanks{Agradecimiento}}
\date{27/02/2021}
%Cuerpo
\begin{document}
	\maketitle
	\tableofcontents %índices
	Este es nuestro primer documento. Durante \today \ aprenderemos documentos de Latex.\date{27/02/2021}
	\textbf{Negrita}
	\underline{Subrayado}
	\textbf{\textit{Negrita y cursiva}}	
	
	\includegraphics{OIP}
	
	
	{\Huge{Texto grande}}
	{\tiny{\textit{texto}}}
	There's a picture of a galaxy above
	\ %espacio
	\\ %Salto de línea
	
	\begin{center}
		Texto centrado...
	\end{center}
	
	\begin{flushleft}
		Texto a la izquierda...
	\end{flushleft}
	
	Hola caracola \hfill texto alineado con espacio a la derecha
	\footnote{Nota de pie de página}
	
	\newpage
	\setcounter{footnote}{1}
	\noindent La \textbf{teoría de juegos}\footnote{ver Casas Méndez, B.(2012). Introducción a la teoría de juegos} es un área de las \underline{matemáticas} que utiliza modelos para estidiar interacciones entre \textit{jugadores} en estructuras formalizadas de incentivos. Existen, principalmente, dos tipos de juegos.\\
	\begin{center}
		{\large{\textbf{Juegos estratégicos \& cooperativos}}}\\
	\end{center}
	
	\noindent Sus investigadores estudian las \ \ \ \ \ estrategias \\ óptimas de cada jugador.\\
	Este documento ha sido creado \dotfill \today
	 
	%\chapter{Capítulo} no se puede por que estamos en un artículo
	\section*{Sección} %*para que no salga en el índice
	\section{Juan Carlos se porta mal}
	\subsection{Juan Carlos se porta ?}
	\subsubsection{Clase de Latex}
	\subsubsection[Cap2]{Subsubsección}
	
	%Listas sin numerar
	\begin{itemize}
		\item Elemento 1 \begin{itemize}
			\item Subelemento 1
			\item Subelemento 2
		\end{itemize}
		\item Elemento 2
	\end{itemize}
	
		%Listas sin numerar
	\begin{enumerate}
		\item Elemento 1 \begin{enumerate}
				\item Subelemento 1
				\item Subelemento 2
			\end{enumerate}
		\item Elemento 2
		\item[4.] Elemento 4
		\setcounter{enumi}{10}
		\item Elemento Enum %11
	\end{enumerate}
	\newpage
	%Ejercicio
	\begin{enumerate}
		\item Primer punto
		\item Segundo punto
		\begin{itemize}
			\item Primer punto del segundo nivel
			\begin{description}
				\item[Tercer nivel 1] Primer punto
				\item[Tercer nivel 2] Segundo punto
			\end{description}
			\item Segundo punto del segundo nivel
			\item Tercer punto del segundo nivel
		\end{itemize}
		\item Tercer punto
		\begin{itemize}
			\item Cuarto punto del segundo nivel
			\item Quinto punto del segundo nivel
		\end{itemize}
		\item[6.] Cuarto punto
		\item[5.] Quinto punto
	\end{enumerate}
	
	
\end{document}