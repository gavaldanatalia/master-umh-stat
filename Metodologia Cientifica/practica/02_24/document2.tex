\documentclass[a4paper, 10pt]{article}
\usepackage[utf8]{inputenc}
\usepackage[spanish]{babel}
\usepackage{lipsum}
\usepackage{graphicx}
\usepackage{hyperref}
\usepackage[all]{hypcap}
\usepackage{color}
\definecolor{color1}{RGB}{19,120,250}
\title{Clase de Metodología Científica} %Preámbulo
\author{Alfonso \and Pepe \thanks{Agradecimiento a Pepe}}
\begin{document}
	\tableofcontents
	\listoffigures
	\listoftables
	\maketitle
	%Texto aleatorio
	\noindent\today \\
	\lipsum[1] \\ \\
	\lipsum[2] \\ \\ \\ \\
	
	\begin{table}[h]
		\label{Tabla 1}
		\begin{center}
			\begin{tabular}{c|cc}
				3 & 1 & 2 \\
				1 & 2 & 3 \\
			\end{tabular}
		\end{center}
	\end{table}

	\begin{table}[h]
		\begin{center}
			\begin{tabular}{c|r|l|c} % 4 columnas centradas %r derecha %l izquierda % center
				 Alumn@ & Nota & Asignatura & Carrera \\ \hline \hline %barra horizontal	 % vertical vline		
				 María & 9 & Estadística & \\
				 Víctor & 7 & Metodología & \\
				 Laura & 8 & Álgebra & \\ \hline
			\end{tabular}
			\caption{Notas del alumnado de primer curso.}
		\end{center}
	\end{table}
	En la Tabla 1 \ref{Tabla 1} se puede ver 
	
	\begin{figure}
		\label{Figura 1}
		\begin{center}
			\includegraphics{OIP} %Ruta de la imágen en la misma carpeta
		\caption{Comandos de Látex para modificar tamaño de texto}
		\end{center}
	\end{figure}
	En la Figura 1 \ref{Figura 1} se puede ver \\
	\textcolor{red}{Texto rojo}
	\colorbox{red}{Texto rojo}
	\fcolorbox{red}{blue}{Texto azul y  rojo}
	
	\colorbox{color1}{Texto random}
	
	%Texto técnico
	$\frac{3}{5}$
	
	$$\frac{3}{5}$$ %Línea aparte
	
\end{document}