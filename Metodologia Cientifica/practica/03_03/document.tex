\documentclass{article}
\usepackage[utf8]{inputenc}
\usepackage{amsmath}

\title{Trabajo en clase}
\author{Natalia Gavaldá Lizán}
\date{3 de marzo de 2025}

\begin{document}
	
	\maketitle

	
	Hola caracola $\displaystyle U_1 = a^{x+y}$
	Esta es mi primera funcion en latex $U_1 = a^{x+y}_2$
	Esta es mi primera funcion en latex: 
		$$U_1 = a^{x+y}_2$$
	
	\section*{Este es mi primer ejercicio}
	
	Una serie $\sum_{n=1}^\infty a_n$ converge si existe un valor $\alpha$ tal que la sucesión de sumas parciales de dicha serie \textbf{converge} a $\alpha$
	
	\[
	\lim_{k \to \infty} \sum_{n=1}^{k} a_n = \lim_{k \to \infty} S_k = \alpha.
	\]
	
	Una serie \textit{converge condicionalmente} si la serie $\sum_{n=1}^\infty a_n$ es convergente pero $\displaystyle \sum_{n=1}^\infty |a_n|$ diverge.
	
	El valor de la integral es
	
	\[
	\int_a^b f(x)\, dx = \lim_{n \to \infty} \frac{b-a}{n} \sum_{k=1}^n f\left(a + k \frac{b-a}{b}\right).
	\]
	
	La raíz es igual a: $ \sqrt{\prod_{i=1}^n a_i} = 
	\displaystyle \prod_{i=1}^n \left[ \sqrt{a_i} \right].$
	
	\section*{Este es mi segundo ejercicio}
	
	$\left(
	\begin{array}{ccc}
		1 &  \cdots & 1 \\
		\vdots & \ddots  & \vdots \\
		1 & \cdots & 1 \\
	\end{array}
	\right)$
	
\end{document}